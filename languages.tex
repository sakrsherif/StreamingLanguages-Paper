\section{Stream Processing Languages}\label{sec:languages}

\begin{alltt}TODO\scriptsize ~2.8 pages total
- this section will have ~0.3 pages about each language, and
  each of these snippets will be structured around questions:
  why-who-when-what-where-whence
  (see CQL for an example, but the others will be similar!)
- before the snippets on each of the languages, we will briefly
  introduce and explain these questions, as follows:
  - why: objective, audience, domain
  - who: inventors, supporters
  - when: first release / first paper
  - what: key idea, data model, type system, code example
  - where: being developed or offered today, license
  - whence: influenced-by and influences
- descriptions of individual languages
\end{alltt}

\begin{figure}[!h]
\begin{lstlisting}
SELECT IStream(Max(len) AS mxl,
                 MaxCount(len) AS num,
                 ArgMax(len, caller) as who)
FROM Calls[Range 24 Hours Slide 1 Minute]
\end{lstlisting}
\vspace*{-4mm}
\caption{\label{fig:cql}CQL code example.}
\end{figure}

\textbf{Relational Streaming.}
In 2004, Arasu et al.\ at Stanford introduced CQL (for Continuous
Query Language)~\cite{arasu_widom_2004}. CQL has been designed as an
SQL-based declarative language for implementing continuous queries
against streams of data, such as the LinearRoad
benchmark~\cite{arasu_et_al_2004}. The design was influenced by the language of
the TelegraphCQ system that proposed a declarative language with a
particular focus on expressive windowing
constructs~\cite{chandrasekaran_et_al_2003}. The semantics of CQL are
based on two data types: \emph{streams} and \emph{relations}. It
supports three classes of operators over these types. First,
\emph{stream-to-relation} operators, which produce a relation from a
stream.  The design of these operators is based on the concept of a
\emph{window} over a stream, which, at any point of time, contains a
historical snapshot of a recent portion of the stream. CQL includes
time-based and tuple-based windows, both with optional
partitioning. Second, \emph{relation-to-relation} operators, which
produce a relation from other relations. These operators are derived
from traditional relational algebra and are expressed using standard
SQL. Third, \emph{relation-to-stream} operators, which produce a
stream from a relation. CQL supports three operators of this class:
IStream, DStream, and RStream (to capture inserts, deletes, or the
relation).  Figure~\ref{fig:cql} illustrates a CQL code example that
uses a time-sliding window (per minute within the last 24 hours) over
phone calls to return the maximum phone call length along with its
count and caller information. CQL has influenced the design of many
systems such as Microsoft StreamInsight~\cite{ali_et_al_2009}.


\begin{figure}[!h]
\begin{lstlisting}
node tracker (speed, limit: int) returns (t: int);
var x: bool; cpt: int when x;
let
  x = (speed > limit);
  cpt = counter((0, 1) when x);
  t = current(cpt);
tel
\end{lstlisting}
\vspace*{-4mm}
\caption{\label{fig:lustre} Lustre code example.}
\end{figure}

\textbf{Synchronous Dataflow.}
Dataflow synchronous languages were introduced to ease the design of
real-time embedded systems. They allow to write a well-defined
deterministic specification of the system. It is then possible to
test, verify, and generate embedded code.
The first dataflow synchronous languages Lustre~\cite{lustre_1987}
(Caspi and Halbwachs) and Signal~\cite{signal_1991} (Le Guernic,
Benveniste, and Gautier) were proposed in France in the late 80s.
A dataflow synchronous program is a set of equations defining streams
of values. Time proceeds by discrete logical steps, and at each step,
the program computes the value of each stream depending on its inputs
and possibly previously computed values.
This approach is reminiscent of block diagrams, a popular notation to
describe control systems.
Figure~\ref{fig:lustre} presents a Lustre code example that tracks the
number of times the speed of a vehicle exceeds the speed limit. The
counter \lstinline{cpt} starts with~$0$ and is incremented by~$1$ each
time the current speed exceed the current limit (\lstinline{when x}).
The return value \lstinline{t} maintain the last computed value
of \lstinline{cpt} between two occurrences of~\lstinline{x}
(\lstinline{current(cpt)}).
The dataflow synchronous approach has inspired
multiple languages: Lucid Synchrone~\cite{lucid_2006} combines the
dataflow synchronous approach with functional features \`a la ML,
StreamIt~\cite{streamit_2002} focuses on efficient processing of large
streaming applications, Z\'elus~\cite{zelus_2013} is a Lustre-like
language extended with ordinary differential equations to define
continuous-time dynamics. Lustre is also the backbone of the
industrial language and compiler Scade~\cite{scade_2017} routinely
used to program embedded controllers in many critical applications.

\begin{figure}[!h]
\begin{lstlisting}
stream<float64 len, rstring caller> Calls = CallsSrc() { }
type StatsT = tuple<float64 len, int32 num, rstring who>;
stream<StatsT> StatsS = Aggregate(Calls) {
   window Calls: sliding, time(24.0*60.0*60.0), time(60.0);
   output StatsS: len = Max(Calls.len),
                  num = MaxCount(Calls.len),
                  who = ArgMax(Calls.len, Calls.caller);
}
\end{lstlisting}
\vspace*{-4mm}
\caption{\label{fig:spl}SPL code example.}
\end{figure}

\textbf{Big-Data Streaming.}
\begin{alltt}TODO\scriptsize, ~0.3 pages, \textcolor{red}{Martin, Emanuele}
- why: scale, and diverse data and processing needs
- who: SPL~\cite{hirzel_schneider_gedik_2017}, Hirzel et al., IBM
- when: 2010
- what: Figure~\ref{fig:spl}, explicit graph of streams and operators,
  operators defined by users or in library (not built-in),
  streams carry tuples, tuples nest other tuples, lists, maps
  - for precise semantics see core calculus~\cite{soule_et_al_2016}
- where: IBM Streams product
- whence: Borealis is evolutionary link from relational \cite{abadi_et_al_2005},
  followed by Storm~\cite{toshniwal_et_al_2014}, Spark Streaming~\cite{zaharia_et_al_2013}
\end{alltt}

\textbf{Complex Event Processing.}
\begin{alltt}TODO\scriptsize, ~0.3 pages, \textcolor{red}{Angela}
- MATCH-RECOGNIZE \cite{zemke_et_al_2007}, MatchRegex \cite{hirzel_2012}
\end{alltt}

\begin{figure}[!h]
\begin{lstlisting}
CREATE ~\textit{cq\_name}~
         ~\textit{xml\_ql\_query}~
     DO ~\textit{action}~
  {START ~\textit{start\_time}~}
  {EVERY ~\textit{time\_interval}~}
  {EXPIRE ~\textit{expiration\_time}~}
\end{lstlisting}
\vspace*{-4mm}
\caption{\label{fig:Niagra}NiagaraCQ code example.}
\end{figure}

\textbf{XML Streaming.}
Chen et al.\ introduced NiagaraCQ~\cite{chen_et_al_2000} as a
continuous query sub-system of the Niagara internet query engine, a
database for querying distributed XML data sets developed at
University of Wisconsin and Oregon Graduate
Institute~\cite{naughton2001niagara}. NiagaraCQ implements continuous
query processing over XML files by supporting incremental evaluation
and considering only the changed portion of each updated XML file. It
supports two types of continuous queries that differ in the criteria
triggering their execution: \emph{change-based} queries, which trigger
as soon as new relevant data becomes available, and \emph{timer-based}
queries, which trigger only at specified time intervals.  NiagaraCQ is
based on XML-QL~\cite{deutsch1999query}.  It provides a command
language for creating continuous queries that follows the form
illustrated in Figure~\ref{fig:Niagra}. Using this command language,
users can implement continuous queries that combines an ordinary
XML-QL query with additional time information.  The
\textsf{\small\textit{xml\_ql\_query}} becomes effective at the
defined \textsf{\small\textit{start\_time}}.  The
\textsf{\small\textit{time\_interval}} indicates how often the query
will be executed. A query is timer-based if its
\textsf{\small\textit{time\_interval}} is non-zero; otherwise, it is
change-based.  The continuous query will be deactivated after its
\textsf{\small\textit{expiration\_time}}. The
\textsf{\small\textit{action}} triggers once the results of the XML-QL
query expression is returned.  A key optimization mechanism used in
the NiagaraCQ system is grouping the execution of similar queries to
minimize redundant work.  YFilter~\cite{diao_et_al_2002} implements
continuous queries based on XPath~\cite{clark_derose_1999}, also with
multi-query optimization. The main idea of YFilter is to use a single
finite state machine to represent and evaluate several XPath
expressions.

\textbf{RDF Streaming.}
\begin{alltt}TODO\scriptsize, ~0.3 pages, \textcolor{red}{Emanuele, Akrivi}
- C-SPARQL \cite{barbieri_et_al_2009}
- stream reasoning
\end{alltt}

\textbf{Reactive Programming.}
\begin{alltt}TODO\scriptsize, ~0.3 pages, \textcolor{red}{Martin}
- ActiveSheets \cite{vaziri_et_al_2014}
\end{alltt}

\textbf{Summary.}
\begin{alltt}TODO\scriptsize
- close with comparison table
- perhaps according to performance/generality/productivity
\end{alltt}
