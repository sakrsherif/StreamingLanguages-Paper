\subsection{Streaming for End-Users}\label{sec:eup} % Martin

\begin{figure}[!h]
\centerline{\includegraphics[width=\columnwidth]{CallStats.jpg}}
\vspace*{-4mm}
\caption{\label{fig:activesheets}ActiveSheets example.}
\end{figure}

We use the term \emph{end-users} to refer to users without particular
software development training. Probably the most successful
programming tool for end-users is spreadsheet formulas. And from the
early days of VisiCalc in 1979~\cite{bricklin_frankston_1979},
spreadsheet formulas have been \emph{reactive} in the sense that any
changes in their inputs trigger an automatic recomputation of their
outputs. Therefore, in 2014, Vaziri et al.\ designed ActiveSheets, a
spreadsheet-based stream programming model~\cite{vaziri_et_al_2014}.
Figure~\ref{fig:activesheets} gives an example that implements a
similar computation as Figures \mbox{\ref{fig:cql} and \ref{fig:spl}}.
Cells \lstinline{A3:B8} contain a sliding window of recent call
records, which ActiveSheets updates from live input data. Cells
\lstinline{D6:F6} contain the output data, \mbox{(re-)}com\-pu\-ted
using reactive spreadsheet formulas. The formula
\mbox{\lstinline{E6=COUNTIF(A3:A8,D6)}} counts how many calls in the
window are as long as a longest call. The formula
\mbox{\lstinline{F6=INDEX(B3:B8,F2)}} uses the relative index \lstinline{F2}
of the longest \lstinline{len} to retrieve the corresponding
caller.  ActiveSheets was influenced by
synchronous data\-flow~\cite{lustre_1987} and has been extended with
time-based windows, key-based partitions, and performance
optimizations~\cite{hirzel_et_al_2016}. Other ideas besides
spreadsheets that have been used to let end-users build streaming
applications include trigger-action programming~\cite{ifttt} and
controlled natural language~\cite{arnold_et_al_2016}.
