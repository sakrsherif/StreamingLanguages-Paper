\subsection{Complex Event Processing}\label{sec:cep} % Angela

%/* Either from similar domain as CQL example:
%  alert when calls from locations over
%  10 miles apart within 60 seconds. */

\begin{figure}[!h]
\begin{lstlisting}
stream<Alert> Alerts = MatchRegex(Calls) {
  param
    partitionBy: caller;
    pattern    : ".+ tooFarTooFast";
    predicates : {
      tooFarTooFast =
            geoDist(First(loc), Last(loc)) >= 10.0
        && timeDist(First(ts), Last(ts)) <= 60.0; };
  output
    Alerts: who=caller, where=Last(loc), when=Last(ts);
}
\end{lstlisting}
\vspace*{-4mm}
\caption{\label{fig:cep}CEP example.}
\end{figure}

%/* Or from similar domain as Lustre example:
%   simple edge detector, each time a car's speed
%   increases over the speed limit, emit a tuple
%   with a count of 1. */
%stream<Edge> Edges = MatchRegex(CarUpdates) {
%  param  pattern    : "below above";
%         partitionBy: car;
%         predicates : { below = speed <= limit;
%                        above = speed > limit; };
%  output Edge: car=car, speed=Max(speed), count=1;
%}

An alternative  technology with respect to stream processing is Complex Event Processing (CEP), which is oftentimes
considered as a special case of stream processing. In particular, the latter consideration has led to define suitable CEP
operators in classical streams processing languages. A prominent example on top of SPL~\cite{hirzel_schneider_gedik_2017} described above is the
MatchRegex~\cite{hirzel_2012} operator, whose implementation is available in the System S distributed streaming platform. This operator has been introduced by Hirzel in 2012 and influenced by the MATCH-RECOGNIZE~\cite{zemke_et_al_2007} version, which is part of the SQL ANSI
standard. With respect to its SQL counterpart,  MatchRegex is much more simplified and concise in terms of syntax as well as easy to deploy as a library operator. The MatchRegex~\cite{hirzel_2012} operator is implemented via code generation and translates to an automaton, which permits space and time-efficient incremental computation of aggregates. 
Other functionalities beyond pattern matching are not supported, such as join operations and reporting tasks.
Figure~\label{fig:cep} shows an example of SPL source code with MatchRegex operator.
%It permits to capture a composite event in which the speed of a vehicle already overcoming the limit rises beyond the maximum value of the speed
%observed so far (corresponding to the first predicate rise). This event can be repeated several times due to the presence of the Kleene-plus in the regular expression of the pattern. Similarly,
%the minimum speed registered within the speed bound is captured in the second predicate and its value is provided as output, along with the number of times when these rises and drops occurs and the first
%foreseen event of either kind. 
It permits to capture a composite event in which the calls issued from locations over $10$ miles apart and whose durations is within $60s$ trigger 
a stream of alerts. This event can be repeated several times due to the presence of the Kleene-plus in the regular expression of the pattern. 
The location and duration are captured in the corresponding predicates and their values are provided as output, along with the corresponding alert. 
We now turn to discuss two close predecessors of MatchRegex, namely the MATCH-RECOGNIZE~\cite{zemke_et_al_2007} clause of SQL and 
~\cite{WuDR06}. The former has been introduced in order to capture pattern recognition in rows of a table. Patterns are described by a regular 
expression included in the PATTERN argument of the MATCH-RECOGNIZE clause and represented by variables. For instance, in our example of Figure~\ref{fig:cep}, the PATTERN argument contains a group variable given by $A+$ due to the presence of the Kleene-plus quantifier (where $A$ is defined AS
LAST(G.Loc) - FIRST(G.Loc) >= 10.0
        AND Last(T.ts) - FIRST(T.ts) <= 60.0 with G and T indicating the geodistance and timestamp relations). The clause permits to specify the output as a singleton (ONE ROW PER MATCH, also the default) or as a set of rows (ALL ROWS PER MATCH). The first option makes visible the columns specified in   the partition
columns, order by columns and columns defined in the MEASURES clause. For ALL
ROWS PER MATCH the visible columns include also all columns of the initial tables even though they are not used in the previous fields.
Since MATCH-RECOGNIZE is embedded in SQL, join conditions can be readily specified. 
With the advent of Big Data, handling higher volumes of streaming data and being capable of accommodating larger windows become imperative. To address these two challenges, the SASE event system proposes to execute complex event queries on real-time streams. 
Figure~\ref{fig:SASE} shows the high-level syntax of the language, where the first clause EVENT specifies the types of events that need to be captured, while the second clause WHERE allows to filter those events by imposing suitable predicates applied to the events attributes and the third clause WITHIN indicates a time period in a sliding window over the input stream.  The first clause also allows to specify a sequence of types of events or their non occurrence 
via the special symbol $!$. However, it is not permitted to specify complex patterns under the form of regular expressions (thus with Kleene-star or Kleene-plus operator) and aggregates. Thus, the above example in Figure~\label{fig:cep} is not expressible in SASE. The idea behind SASE is to have at disposal 
a self-contained language which can be translated to suitable algebraic query expressions. 

\begin{figure}[!h]
\begin{lstlisting}
EVENT ~\textit{event\_pattern}~
[WHERE ~\textit{qualification}~]
[WITHIN ~\textit{window}~]
\end{lstlisting}
\vspace*{-4mm}
\caption{\label{fig:SASE}Syntax of the Sase CEP Language.}
\end{figure}
