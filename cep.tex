\subsection{Complex Event Processing}\label{sec:cep} % Angela

\begin{figure}[!h]
\begin{lstlisting}[morekeywords={stream,param,output}]
stream<Alert> Alerts = MatchRegex(Calls) {
  param
    partitionBy: caller;
    predicates : {
      tooFarTooFast =
            geoDist(First(loc), Last(loc)) >= 10.0
        && timeDist(First(ts), Last(ts)) <= 60.0; };
    pattern    : ".+ tooFarTooFast";
  output
    Alerts: who=caller, where=Last(loc), when=Last(ts);
}
\end{lstlisting}
\vspace*{-4mm}
\caption{\label{fig:cep}CEP example.}
\end{figure}

Complex event processing (CEP) uses patterns over simple events to
detect higher-level, \emph{complex}, events that may comprise multiple
simple events.  CEP can be considered either as an alternative to
stream processing or as a special case of stream processing. The
latter consideration has led to the definition of CEP operators in
streaming languages. For example, the \textsf{MatchRegex}~\cite{hirzel_2012}
operator implements CEP in the library of the SPL
language~\cite{hirzel_schneider_gedik_2017} (Section~\ref{sec:big}). MatchRegex was introduced by Hirzel in~2012,
influenced by the \mbox{\textsf{M{\small{}ATCH}-R{\small{}ECOGNIZE}}} proposal for extending ANSI
SQL~\cite{zemke_et_al_2007}.  Compared to its SQL counterpart,
MatchRegex is simplified, syntactically concise, and easy to deploy as
a library operator. MatchRegex is implemented via code generation and
translates to an automaton for space- and time-efficient incremental
computation of aggregates. However, it omits other
functionalities beyond pattern matching, such as joins and reporting
tasks. Figure~\ref{fig:cep} shows an example for detecting a complex
event when simple phone-call events occur over 10~miles apart within
60~seconds. Line~8 defines the regular expression, where the period
(\lstinline{.}) matches any simple event; the plus (\lstinline{+})
indicates at-least-once repetition; and \lstinline{tooFarTooFast} is a
simple event defined via a predicate in \mbox{Lines 5--7}. The
\lstinline{First} and \lstinline{Last} functions reference
corresponding simple events in the overall match: in this case, the
start of the sequence matched by \lstinline{.+} and the simple event matched by
\lstinline{tooFarTooFast}.

%% 2006
One of the earliest languages for complex event queries on real-time
streams was \textsf{SASE}~\cite{WuDR06}. The language was designed to translate
to algebraic operators, but did not yet support aggregation or regular
expressions with Kleene closure, as used in Figure~\ref{fig:cep}.
%% 2007
The Cayuga Event Language offered aggregation and Kleene closure, but
did so in a hand-crafted syntax instead of familiar regular expression
syntax~\cite{demers_et_al_2007}.
%% 2007
The closest predecessor of MatchRegex was the \textsc{Match-Recognize}
proposal for extending SQL with pattern recognition in relational
rows~\cite{zemke_et_al_2007}. It used regular-expression syntax as
well as aggregations. Like MatchRegex, it is embedded in a host
language that supports orthogonal features via operators such as
joins.
%% 2008
Another take on CEP using regular expressions was
\textsf{EventScript}~\cite{cohen_kalleberg_2008}, which allowed the patterns to
be interspersed with action blocks.
%% 2010
Most CEP pattern matching is inherently sequential, but Chandramouli et
al.\ explored how to generalize it for out-of-order data
streams~\cite{chandramouli_goldstein_maier_2010}. Their work
anticipates the veracity challenge, further discussed in
Section~\ref{sec:veracity} of this survey.

%new part on support of CEP in Microsoft Trill, Apache Flink and Esper
% we can remove the previous statement in case of lack of space

Recently, CEP is also supported by several
high-performance commercial streaming engines, such as Microsoft
Trill~\cite{chandramouli_et_al_2014}, Esper~\cite{Esper18} and Apache
Flink~\cite{carbone_et_al_2015}, the latter
exhibiting a CEP library since its early 1.0 version. 
Indeed, the high throughput
and low latency nature of these engines make them suitable for CEP's real
time analytics. 
