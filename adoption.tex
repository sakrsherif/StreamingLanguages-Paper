\subsection{Adoption}\label{sec:adoption} % Martin

\begin{alltt}TODO\scriptsize ~1 page
- what's the problem
  - as Section~\ref{sec:languages} illustrates, many languages in
    many families, but none broadly adopted
  - out of the language families, the big-data ones (Section~\ref{sec:big})
    are collectively most broadly adopted, but most of them are
    not stand-alone languages but libraries ("embedded languages")
  - that makes them hard to define in isolation and limits their
    adoption to that of their host language
  - challenge: getting broad adoption for a streaming language
- why is solving it useful
  - when streaming languages become adopted more,
    streaming systems will too
  - students can direct their learning efforts
  - companies have more skilled candidates for hiring
  - researchers can direct their innovation
  - benchmark results become easier to compare across systems,
    leading to (hopefully healthy) competition
    - on the other hand, broad adoption of one language leads to
      a a less diverse eco-system, which can reduce competition
  - systems become more robust and faster
- why is it hard
  - if it were easy it would have happened by now, but it has not
  - because of interoperability and expediency, several big-data
    languages are embedded in a host language
  - but broad adoption benefits from stable and clear semantic
    foundation, as can be seen from SQL and relational databases
  - embedded languages tend to lack stable and clear semantics,
    because entangled with host language
  - no agreement on the most important features to include,
    for instance, in the areas of veracity and variety,
    discussed above
  - language adoption not driven just by the technical merits of
    the language, but by external factors, such as industry
    support, or open-source with open governance
- what makes it a streaming languages problem
  - while streaming in general is not new, big-data streaming is
    relatively new
  - the industry trends that drive big-data streaming, including
    IoT, cloud, and AI, are all actively shifting
  - streaming research itself is also actively shifting
  - in some of the industry trends (e.g. AI), innovation starts
    first in an at-rest setting, and then later gets transferred
    to the in-motion setting
  - people come to streaming research from different perspective,
    and may not be familiar with each other's work; this survey
    aims at mitigating that
\end{alltt}
