\section{What's Next?}\label{sec:whatsnext}

\begin{alltt}TODO\scriptsize ~3 pages total
- introductory paragraph
\end{alltt}

\subsection{Veracity}\label{sec:veracity}

%Challenge: Handling veracity in a simple and well-defined way

%Veracity: English definition: Conformance to the facts. Accuracy.

With the evolution of the internet of things and related technologies,
many end-user applications require stream processing and analytics.
Streaming languages should ensure veracity of the output stream in
terms of accuracy, correctness, and completeness of the results.
Furthermore, they should not sacrifice performance either, answering
high-throughput input streams with low-latency output streams.
Veracity in a streaming environment depends on the semantics of the
language since the stream is infinite and new results may be added or
computed aggregates may change. It is important that the output stream
for a given input stream be well-defined based on the streaming
language semantics. For example, if the language offers a sliding
time window feature, any aggregate should be computed correctly at
any time point based on all data within the time window.
%
Stream veracity problems may occur for different reasons. For example,
in multi-streaming applications, each stream may be produced by
sensors. Errors may occur either in the data itself (e.g., noisy
sensor readings) or by delays or data loss during the transfer to the
stream processing system. For instance, data may arrive
out-of-order because of communication delays or because of the
inevitable time drift between independent distributed stream sources.
Ideally, the output stream should be accurate, complete, and timely
even if errors occur in the input stream.  Unfortunately, this is not
always feasible.

\textbf{Why is this important?}
%
Veracity of the output of streaming applications is important when
high-stakes and irreversible decisions are based on these outputs.  In
the big-data era, veracity is one of the most important problems even
for non-streaming data processing, and stream processing makes
veracity even more challenging than in the static case. Streams are
dynamic and usually operate in a distributed environment
with minimal control over the underlying infrastructure. Such a
loosely coupled model can lead to situations where any data source can
join and leave on the fly. Moreover, stream-producing sensors
have limitations such as processing power, energy level, or memory
consumption, which can easily compromise veracity.
 %including data incompleteness, lack of quality and/or concept drifts. Additionally, in a dynamic environment where data is evolving over time and can change its properties and statistics, which demand novel veracity handling algorithms compared to static datastores.  

%Ability of the stream management systems to handle veracity will facilitate handling robust application design and ensure correctness of results. With the evolution of internet of things and relevant technologies, a large number of applications are designed for end users which require real-time stream processing and analytics on the fly such as applications designed in the domains of IoT, healthcare and data on the Web. 
%
%Particularly for stream mining, the traditional data mining algorithms which focus on batch processing to improve quality of data are not enough to cope with the veracity challenge while meeting the time constraints of stream processing. This is why it is important to have proper streaming methods of data mining and machine learning to process streaming in real-time, since the batch methods cannot be applied in this setting.
%
%Benchmarking is a very important aspect of any data processing systems to ensure correctness, validation, testing and reproducibility. Traditional benchmarking systems focus mainly on performance evaluation, but since stream processing brings a few additional dimensions including veracity of data. Hence, benchmarking systems designed for testing stream processing systems should cover a broad range of parameters including veracity. 

\textbf{How can we measure the challenge?}
%
To estimate the robustness of a streaming language implementation to
veracity problems, we define as \emph{ground truth} the output stream
in the absence of veracity problems (for example data loss or delayed
data). Then we can quantify veracity. Let \emph{error} be a function
that compares the produced result of an approach with and without
veracity problems.  An example of an error function is the number of
false positives and false negatives.  An approach is \emph{robust} for
veracity of streaming data if the error scales at most linearly with
respect to the size and the error rate of the input stream, while the
delay in the latency is bounded and independent of the input size.
%
The streaming language veracity challenge can be broken down
into the following measures $\mathbf{C_1}$--$\mathbf{C_3}$:

\begin{itemize}
  \item[$\mathbf{C_1}$] \emph{Fault-tolerance.} A program in the
    language is robust even if some of its components fail. The
    language can define different behaviors, for example,
    at-least-once semantics in Storm~\cite{toshniwal_et_al_2014}
    or check-pointing in Spark Streaming~\cite{zaharia_et_al_2013}.
  \item[$\mathbf{C_2}$] \emph{Out-of-order handling.} This measure has
    two facets. First, the streaming language should have clear
    semantics about the expected result. Second, the streaming language should be
    robust to out-of-order data and should ensure that the expected
    output stream is produced with limited latency. Li et al.\ define
    out-of-order stream semantics based on low
    watermarks~\cite{Li:2008:OPN:1453856.1453890}; Spark Streaming
    relies on idempotence to handle stragglers~\cite{zaharia_et_al_2013}; and Beam separates
    event time from processing time~\cite{akidau_et_al_2015}.
  \item[$\mathbf{C_3}$] \emph{Inaccurate value handling.} A program in
    the language is robust even if some of its input data is wrong.
    The language can help by supporting statistical quality
    measures~\cite{wasserkrug_et_al_2008}.
\end{itemize}

%%Veracity refers to several data quality dimensions, such as accuracy, conformance, completeness, correctness and timeliness. When shifting from a static or batch setting to a streaming one, these dimensions need a different interpretation. We can classify the aspects of veracity into two broad categories: those related to the data itself and those dependent of the underlying stream processing. The inherent characteristics of streaming data such as the unboundedness, the uncertainty, the incompleteness and the non-deterministic order are definitely more challenging in the case of streams with respect to static data. Where the unboundedness and non-deterministic order are tied to the streaming setting, the uncertainty and the incompleteness are also crucial in the static case. The latter are however exacerbated in the presence of streams. On the other hand, timeliness which refers to the freshness of the data is more guaranteed in a streaming system than in a static environment. 
%%One open problem is to come up with algorithms and techniques to check the validity of the streaming data in an online fashion. One obstacle to address is the lack of ground truth (or, gold standard) and the partial availability of the ground truth (if we assume to build the ground truth incrementally as new streaming data come)
%%
%%
%%. 
%%Veracity of streaming data is also affected by the results of streaming processing algorithms, such as the execution of online algorithms, approximate computing techniques, stream mining and learning algorithms.  These algorithms may produce new streams that are inherently less accurate and trustworthy than the original data. One additional challenge here is to design streaming processing algorithms that are quality-aware.  These streaming processing algorithms aiming at addressing the veracity problem should be self-adaptive and configurable to accommodate new defects and anomalies in the upcoming data. 
%%
%%
%%
%%Veracity shall be handled at the level of data instances and at the level of stream processing algorithms. Data has inherently different degrees of veracity (out of order, data loss, delay in the data delivery etc.) and checking the validity of streaming data poses new challenges than in the static case. As an example, a snapshot-centered definition of validity on the data is needed. 
%%For what concerns the streaming processing algorithms, they bring approximate results that may introduce further veracity problems. How can we ensure that these algorithms do not worsen the veracity degree of the underlying data and lead to minimize the introduced errors and uncertainties? 
%%
%%%% \subsubsection{Metrics}
%%%One major issue is if an stream processing approach is robust to veracity. In order to estimate the robustness of an approach to data lost or data that are out of date we define as ground truth the result of the stream processing in the absence of data loss or data that are delayed. Then we can quantify the error produced by the veracity. Let $error$ be a function that compare the produced result of an approach with and without veracity. An approach is robust to veracity of streaming data if $error$ scale linearly with respect to size of the data and the number of errors. The types of errors of concern in a streaming setting include timeliness of delivery of stream elements, fraction of elements that arrive out-of-order, element loss rate, and element integrity verification failure rate. An example of an $error$ function is the number of false positives and false negatives compared to the ground truth.

\textbf{Why is this difficult?}
%
In stream processing, data is typically sent on a best-effort basis.
As a result, data can be lost, incorrect, arrive out of order, or be
approximate. This is exacerbated by the fact that the streaming
setting affords limited opportunity to compensate for these issues.
Furthermore, the performance requirements of streaming systems
encourage the use of approximate computing~\cite{babcock_et_al_2002},
thus increasing the uncertainty of the data. Also, machine-learning
often yields uncertain results due to imperfect generalization.
An important aspect of streaming data is ordering, typically
characterized by time.  The correctness of the response to
queries depends on the source of ordering, such as the creation,
processing, or delivery time.  Stream processing often requires that
each piece of data must be processed within a window, which can be
characterized by predefined size or temporal constraints. In stream
settings, sources typically do not receive control feedback.
Consequently, when exceptions occur, recovery must occur at the
destination. This reduces the space of possibilities for handling
transaction rollbacks and fault tolerance.


\subsection{Data Variety}

Data variety refers to the presence of different data formats, data
types, data semantics, and associated data management solutions in an
information system. The term emerged with the advent of Big Data, but
the problem of taming variety is well known for machine understanding
of unstructured data such as text, images, and video as well as
(syntactic, structural, and semantic) interoperability and data
integration for structured and semistructured data.  There are
multiple known solutions to data variety for a moderate number of
high-volume data sources.  But data variety is still unsolved when
there are hundreds of data sources to integrate or when the data to
integrate is highly dynamic or streaming (as in this paper).

\textbf{Why is this important?}
%
Increasingly, applications must process heterogeneous data streams in
real-time together with large background knowledge bases. Consider the
following two examples from~\cite{DellAglioDataScience2017} (where interested readers can find others).

In the first example, we want to use sensor readings of the last
10~minutes to find electricity-producing turbines that are in a state
similar (e.g., Pearson correlated by at least 0.75) to any turbine
that subsequently had a critical failure. Here, data variety arises
from having tens of turbines of 3-4 different types equipped with
different sensors deployed other many years, where more sensors will be
deployed in the future. Moreover, in many cases, once an anomaly is
detected, the user also needs to retrieve multimedia maintenance
instructions and annotations to complete the diagnosis process.

In the second example, we want to use the latest open traffic
information and social media as well as the weather forecast to
determine if the users of a mobile mobility app are likely to run into
a traffic jam during their commute tonight and how long it will take
them to get home. Here, data variety arises from using third-party
data sources that are free to evolve in syntax, structure, and
semantics.

%There are many more examples that highlight the opportunity to create
%value by taming variety and velocity simultaneously. In social media
%analytics, it would be valuable to identify the current top
%influencers that are driving the discussion about the top emerging
%topics across all the social networks. In the tourism industry, it
%would be valuable to tell tourist where they can spend their evening
%given the presence of people and what the are doing (predicted
%analyzing the spatio-temporal correlation between privacy-preserving
%aggregates of mobile telecom data and of geo-located social media
%posts). In well-being analytics, it would be worth advising people when
%to go exercise, given their past, possibly sedentary, behavior and
%allergies (accessed in a privacy-preserving manner) as well as current
%weather conditions and pollution/allergen levels.

\textbf{How can we measure the challenge?}
%
The streaming language data variety challenge can be broken down
into the following measures $\mathbf{C_4}$--$\mathbf{C_6}$:

\begin{itemize}[leftmargin=6mm]
  \item[$\mathbf{C_4}$] \emph{Expressive data model.}  The data model
    used to logically represent information is expressive and allows
    encoding multiple data types, data structures, and data
    semantics. This is the path investigated by
    RSP-QL~\cite{DellAglioDataScience2017,DBLP:conf/debs/ValleDM16}.
  \item[$\mathbf{C_5}$] \emph{Multiple representations.} The language
    can ingest data in multiple representations, offering the
    programmer a unified set of logical operators while implementing
    physical operators that work directly on the representations for
    performance. An example is the most recent evolution of the
    Streaming Linked Data framework~\cite{DBLP:conf/esws/BalduiniV017a}.
  \item[$\mathbf{C_6}$] \emph{New sources with new formats.} The
    language allows adding new sources where data are represented in a
    format unforeseen when the language was
    released. This might be accomplished by extending
    R2RML\footnote{\url{https://www.w3.org/TR/r2rml}}.
%     -- a language
%    for expressing customized mappings from relational databases to
%    RDF datasets.
\end{itemize}

\textbf{Why is this difficult?}
%
Deriving value is harder for a system that has to tame data variety
than for a system that only has to handle a single well-structured
data source. This is because solutions that analyze data require
homogeneous well-formed input data, so, when there is data variety,
preparing such data requires a number of different data management solutions that take time to
perform their part of the processing as well as to coordinate among each others. This time is particularly relevant in stream
processing, where answers should be generated with low latency. Even
if the time available to answer depends on the application domain (in
call centers, routing needs to be decided in sub-seconds, while in oil
operations, dangerous situations must be detected within
minutes), traditional batch pipelines for feature extraction and
extract-transform-load (ETL) may take so long that the results, when
computed, are no longer useful. For this reason, it is still
challenging to tame variety in stream processing systems.

\subsection{Adoption}\label{sec:adoption} % Martin

Stream processing languages have an adoption problem. As
Section~\ref{sec:languages} illustrates, there are several families of
streaming languages comprising several families each.  But no one
streaming language has been broadly adopted. The language family
receiving the most attention from large technology companies is
big-data streaming, including offerings by
Google~\cite{akidau_et_al_2013}, Microsoft~\cite{ali_et_al_2009},
IBM~\cite{hirzel_schneider_gedik_2017}, and
Twitter~\cite{toshniwal_et_al_2014}. However, they all differ.
Furthermore, in the pursuit of interoperability and expediency, most
big-data streaming languages are not stand-alone but embedded in a
host language. While being embedded gives a short-term boost to
language development, the entanglement with a host language makes it
hard to offer stable and clear semantics. And, if the history of
databases is any guide, such stable and clear semantics are useful for
agreeing on and consistently implementing a standard. Part of the
reason that the relational model for databases displaced its disparate
predecessors is its strong mathematical foundation.  One of the
most-used languages mentioned in this survey is
Scade~\cite{scade_2017}, but it is designed for embedded systems and
not big-data streaming. Getting broad adoption for a big-data
streaming language remains an open challenge.

\textbf{Why is this important?}
Solving the adoption problem for stream processing languages would
yield many benefits. It would encourage students to build marketable
skills and give employers a sustainable hiring pipeline. From the
perspective of streaming researchers and vendors, increased adoption
of streaming languages would raise attention to streaming research and
streaming products, respectively. If most systems adopted more-or-less
the same language, they would become easier to benchmark against each
other. Other popular programming languages, such as SQL, Java, and
JavaScript, have benefited when companies competed against each other
to provide better implementations of the language.  On the downside,
focusing on a single language would reduce the diversity of the
eco-system, transforming innovation and competition from being broad
to being deep. But overall, if the problem of streaming language
adoption were solved, we would expect streaming systems to become more
robust and faster.

\textbf{How can we measure the challenge?}
The streaming language  adoption challenge can be broken down
into the following measures $\mathbf{C_7}$--$\mathbf{C_9}$:

\begin{itemize}
  \item[$\mathbf{C_7}$] \emph{Widely-used implementation of one
    language.}  One language in the family has at least one
    implementation that is widely used in practice, for instance,
    Scade for SDF~\cite{scade_2017}.
  \item[$\mathbf{C_8}$] \emph{Standard proposal or standard.}  There
    are serious efforts towards an official standard, for instance,
    Jain et al.\ for StreamSQL~\cite{jain_et_al_2008} or
    \textsc{Match-Recognize} for CEP~\cite{zemke_et_al_2007}.
  \item[$\mathbf{C_9}$] \emph{Multiple implementations of same
    language.}  One language in the family has multiple more-or-less
    compatible implementations, for instance,
    Lustre~\cite{lustre_1987} and Scade~\cite{scade_2017} for SDF.
\end{itemize}

Language adoption is driven not just by the technical merits of the
language itself but also by external factors, such as industry support
or implementations that are open-source with open governance.

\textbf{Why is this difficult?}
Adoption is hard for any programming language, but we believe it is
particularly hard for a streaming language. While streaming in general
is not new~\cite{stephens_1997}, big-data streaming is a relatively
recent phenomenon. And big-data streaming, in turn, is driven by
several ongoing industry trends, including the internet of things
(IoT), cloud computing (rentable large-scale elastic compute
resources), and artificial intelligence (AI). Since all three of these
trends are themselves actively shifting, they provide an unstable
environment for streaming languages to grow up in. Furthermore,
innovation often takes place in a setting where data is assumed to be
at rest, as opposed to streaming, where data is in motion. For
instance, most AI algorithms work over a fixed training data set, so
additional research is necessary to make them work well online.  When
it comes to streaming languages, there is not even a consensus on what
are the most important features to include. For instance, both the
veracity and the variety challenge discussed previously have given
rise to many feature ideas that have yet to make it into the
main-stream.  Since people come to streaming research from different
perspectives, they sometimes do not even know each other's work,
inhibiting adoption. This survey aims to mitigate that problem.


\subsection{Challenges Summary}

\begin{table}
  \centerline{\begin{tabular}{@{}lccc@{}}
    \toprule
    \emph{Languages} & \emph{Veracity} & \emph{Variety} & \emph{Adoption}\\
    \midrule
    Relational   &       C$_2$       &                   &       C$_8$ C$_9$\\
    Synchronous  &                   & C$_4$             & C$_7$       C$_9$\\
    Big-data     & C$_1$ C$_2$       & C$_4$ C$_5$ C$_6$ & C$_7$            \\
    CEP          &       C$_2$       & C$_4$             &       C$_8$      \\
    XML          &                   & C$_4$       C$_6$ &                  \\
    RSP-QL      & C$_3$* & C$_4$* C$_5$ C$_6$ & C$_8$ C$_9$\\
    End-user     &                   & C$_4$             &                  \\
    \bottomrule
    \multicolumn{4}{l}{\scriptsize * only when used in combination with Stream Reasoning} 
  \end{tabular}}
  \caption{\label{tab:challenges}Which of the language families from
    Section~\ref{sec:languages} address which of the measures of
    streaming language challenges in Section~\ref{sec:whatsnext}.}
\end{table}

\begin{alltt}TODO\scriptsize \textcolor{red}{Martin}
- explain Table~\ref{tab:challenges}
- of course, Table~\ref{tab:challenges} is more sparse
  than Table~\ref{tab:principles} on coverage of principles
- ultimately, we would want to have language designs that
  are both principled and close the gap on challenges
\end{alltt}
