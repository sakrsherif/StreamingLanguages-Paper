\section{Principles}\label{sec:principles}

The previous section described concrete stream processing languages
belonging to several families. This section takes a cross-cutting view
and explores concepts that many of these languages have in common by
identifying the language design principles behind the concepts.
%(From Angela) For the sake of conciseness, I have removed the two statements below.
%The views and opinions expressed herein are those of the authors and
%are meant to stimulate discussion rather than be the final word.
%Explicitly articulating principles demystifies the art of language
%design and makes it more of a science. 
We categorize language design
principles according to the three requirements from
Section~\ref{sec:introduction}, namely performance, generality, and
productivity.

The \textbf{performance} requirement is addressed by streaming
language design principles $\mathbf{P_1}$--$\mathbf{P_4}$:
\begin{itemize}[leftmargin=6mm]
  \item[$\mathbf{P_1}$] \emph{Windowing principle.} Windows turn streaming
    data into static data suitable for optimized static computation.
    For instance, in CQL, windows produce relations suitable for
    classic relational algebra~\cite{arasu_babu_widom_2006},
    optimizable via classic relational rewrite rules (see
    Figure~\ref{fig:cqlops}).
  \item[$\mathbf{P_2}$] \emph{Partitioning principle.} Key-based partitions
    enable independent computation over disjoint state, thus
    simplifying data parallelism~\cite{schneider_et_al_2015}.
    For instance, MatchRegex performs complex event processing separately by
    partition~\cite{hirzel_2012} (see Line~3 of Figure~\ref{fig:cep}).
  \item[$\mathbf{P_3}$] \emph{Stream graph principle.} Streaming
    applications are graphs of operators that communicate almost
    exclusively via streams, making them easy to place on different
    cores or machines. For instance, SPL programs are explicit stream
    graphs~\cite{hirzel_schneider_gedik_2017} (see
    Figure~\ref{fig:spl}).
  \item[$\mathbf{P_4}$] \emph{Restriction principle.} The schedules and
    communication rates in a streaming application are restricted for
    both performance and safety. For instance, Lustre can be compiled
    to a simple imperative control loop without communication
    buffers~\cite{lustre_1987} (see Section~\ref{sec:sdf}).
\end{itemize}

The \textbf{generality} requirement is addressed by streaming language
design principles $\mathbf{P_5}$--$\mathbf{P_8}$:
\begin{itemize}[leftmargin=6mm]
  \item[$\mathbf{P_5}$] \emph{Orthogonality principle.} Basic language
    features are irredundant and work the same independently of how
    they are composed. For instance, in CQL, relational-algebra
    operators behave the same no matter which windows they are
    composed with~\cite{arasu_babu_widom_2006} (see
    Section~\ref{sec:sql}).
  \item[$\mathbf{P_6}$] \emph{No-built-ins principle.} The core language
    remains slim and regular by enabling extensions in the library. For
    instance, in SPL, relational operators are not built into the
    language, but are user-defined in the library
    instead~\cite{hirzel_schneider_gedik_2017} (see \mbox{Lines 3--8}
    of Figure~\ref{fig:spl}).
  \item[$\mathbf{P_7}$] \emph{Auto-update principle.} The syntax of
    conventional non-streaming computation is overloaded to also
    support reactive computation. For instance, ActiveSheets uses
    conventional spreadsheet formulas, updating their output when
    input cells change~\cite{vaziri_et_al_2014} (see
    Figure~\ref{fig:activesheets}).
  \item[$\mathbf{P_8}$] \emph{General-feature principle.} Similar
    special-case features are replaced by a single more-general
    feature. For instance, operator parameters in
    SPL~\cite{hirzel_schneider_gedik_2017} accept general
    uninterpreted expressions, including predicates for the special
    case of CEP~\cite{hirzel_2012} (see \mbox{Lines 4--7} of
    Figure~\ref{fig:cep}).
\end{itemize}

The \textbf{productivity} requirement is addressed by streaming
language design principles $\mathbf{P_9}$--$\mathbf{P_{12}}$:
\begin{itemize}[leftmargin=6mm]
  \item[$\mathbf{P_9}$] \emph{Familiarity principle.} The syntax of
    non-streaming features in streaming languages is the same as in
    non-streaming languages. This makes the streaming language easier
    to learn. For instance, CQL~\cite{arasu_widom_2004} adopts the
    select-from-where syntax of SQL (see Figure~\ref{fig:cql}).
  \item[$\mathbf{P_{10}}$] \emph{Conciseness principle.} The most concise
    syntax is reserved for the most common tasks. This increases
    productivity since there is less code to write and read. For
    instance, regular expressions represent ``followed-by'' concisely
    via juxtaposition \mbox{$e_1\,e_2$} (see Line~8 of
    Figure~\ref{fig:cep}).
  \item[$\mathbf{P_{11}}$] \emph{Regularity principle.} Data literals,
    patterns that match them, and/or declarations all use similar
    syntax. For instance, RSP-QL uses pattern syntax resembling
    concrete RDF triples (see Line~10 of Figure~\ref{fig:rspql}).
  \item[$\mathbf{P_{12}}$] \emph{Backward reference principle.} The code
    direction is consistent with both scope and control dominance.
    This reduces confusion when reading code. For example, Lustre has
    declarations for parameters, return values, and local variables
    before their use (see Figure~\ref{fig:lustre}).
\end{itemize}

\begin{table}
  \centerline{\small\begin{tabular}{@{}l@{ }c@{ }cc@{}}
    \toprule
    \emph{Language} & \emph{Performance} & \emph{Generality} & \emph{Productivity}\\
    \midrule
    CQL          & $\mathbf{P_1 P_2}$ P$_3$       & $\mathbf{P_5}$             P$_8$
& $\mathbf{P_9}$\\
    Lustre       &                   P$_4$ & $\mathbf{P_5}$ P$_6$ P$_7$ P$_8$
& $\mathbf{P_9}$ P$_{10}$ P$_{11}$ P$_{12}$\\
    SPL          & $\mathbf{P_1 P_2}$ P$_3$       & $\mathbf{P_5}$ P$_6$       P$_8$
& $\mathbf{P_9}$          P$_{11}$ P$_{12}$\\
    MatchRegex   &       $\mathbf{P_2}$             & $\mathbf{P_5}$ P$_6$       P$_8$
& $\mathbf{P_9}$ P$_{10}$          P$_{12}$\\
    YFilter      & P$_4$ & $\mathbf{P_5}$ P$_6$ & $\mathbf{P_9}$ P$_{10}$\\
    RSP-QL       &       $\mathbf{P_1}$ P$_3$    & $\mathbf{P_5}$ P$_6$  P$_8$ &
$\mathbf{P_9}$ P$_{10}$ P$_{11}$         \\
    ActiveSheets & $\mathbf{P_1 P_2}$       P$_4$ & $\mathbf{P_5}$ P$_6$ P$_7$ P$_8$
& $\mathbf{P_9}$ P$_{10}$                  \\
    \bottomrule
  \end{tabular}
  }
  \caption{\label{tab:principles}Which of the languages that served as
    examples in Section~\ref{sec:languages} satisfy which of the
    language design principles in Section~\ref{sec:principles}
    (in bold the most occurring principles per requirement).}
\end{table}

Good language design is driven by principles, but it is also an
exercise in prioritizing among these principles. For instance, CQL
satisfies P$_9$ (familiarity principle) by adopting SQL's syntax and
CQL violates P$_{12}$ (backward reference principle) by adopting SQL's
scoping rules. Table~\ref{tab:principles} summarizes principles by
language. We can observe that among 
all principles only two of them (P$_5$ and P$_9$, respectively) 
are uniformly covered by all languages, while P$_1$ and P$_2$
are the most occurring principles of the performance 
requirement. By opposite, some 
of the languages exhibit less principles overall or less 
principles per requirement, respectively. 
Whereas Table~\ref{tab:principles} is informative in terms of 
boolean coverage of principles per language, it does not 
provide a comparative metric~\footnote{Such a metric would not be trivial
to define.} for quantifying the coverage of each principle.
As a consequence, we observe that satisfying more principles does not automatically
imply satisfying the associated requirement better.

Also, note that while
we formulated the principles from the perspective of streaming
languages, we do not claim to have invented them: many are well-known
from the design of other programming languages. For instance, the
orthogonality principle was a stated aim in the Algol~68 language
specification~\cite{vanwijngaarden_et_al_1975}. Now that we have seen
concepts that are \emph{present} in most streaming languages, the next
section will explore what is commonly \emph{missing or
  underdeveloped}.
