\subsection{Synchronous Dataflow}\label{sec:sdf} % Guillaume

\begin{figure}[!h]
\begin{lstlisting}
node tracker (speed, limit: int) returns (t: int);
var x: bool; cpt: int when x;
let
  x = (speed > limit);
  cpt = counter((0, 1) when x);
  t = current(cpt);
tel
\end{lstlisting}
\vspace*{-4mm}
\caption{\label{fig:lustre} Lustre code example.}
\end{figure}

Dataflow synchronous languages were introduced to ease the design of
real-time embedded systems. They allow to write a well-defined
deterministic specification of the system. It is then possible to
test, verify, and generate embedded code.
The first data\-flow synchronous languages Lustre~\cite{lustre_1987}
(Caspi and Halbwachs) and Signal~\cite{signal_1991} (Le Guernic,
Benveniste, and Gautier) were proposed in France in the late `80s.
A dataflow synchronous program is a set of equations defining streams
of values. Time proceeds by discrete logical steps, and at each step,
the program computes the value of each stream depending on its inputs
and possibly previously computed values.
This approach is reminiscent of block diagrams, a popular notation to
describe control systems.
Figure~\ref{fig:lustre} presents a Lustre code example that tracks the
number of times the speed of a vehicle exceeds the speed limit. The
counter \lstinline{cpt} starts with~$0$ and is incremented by~$1$ each
time the current speed exceed the current limit (\lstinline{when x}).
The return value \lstinline{t} maintain the last computed value
of \lstinline{cpt} between two occurrences of~\lstinline{x}
(\lstinline{current(cpt)}).
The dataflow synchronous approach has inspired
multiple languages: Lucid Synchrone~\cite{lucid_2006} combines the
dataflow synchronous approach with functional features \`a la ML,
StreamIt~\cite{streamit_2002} focuses on efficient processing of large
streaming applications, and Z\'elus~\cite{zelus_2013} is a Lustre-like
language extended with ordinary differential equations to define
continuous-time dynamics. Lustre is also the backbone of the
industrial language and compiler Scade~\cite{scade_2017} routinely
used to program embedded controllers in many critical applications.
