\subsection{Big-Data Streaming}\label{sec:big} % Martin

\begin{figure}[!h]
\begin{lstlisting}[xleftmargin=2mm,morekeywords={stream,float64,rstring,type,int32,window,output,tuple}]
stream<float64 len, rstring caller> Calls = CallsSrc() {}
type Stat = tuple<float64 len, int32 num, rstring who>;
stream<Stat> Stats = Aggregate(Calls) {
   window Calls:sliding, time(24.0*60.0*60.0), time(60.0);
   output Stats: len = Max(Calls.len),
                 num = MaxCount(Calls.len),
                 who = ArgMax(Calls.len, Calls.caller);
}
\end{lstlisting}
\vspace*{-4mm}
\caption{\label{fig:spl}SPL code example.}
\end{figure}

The need to handle diverse data and processing requirements at scale
motivated several recent big-data streaming languages and
systems~\cite{akidau_et_al_2013,carbone_et_al_2015,hirzel_schneider_gedik_2017,toshniwal_et_al_2014,zaharia_et_al_2013}.
Each of them makes it easy to integrate operators written in
general-purpose languages and to parallelize them on clusters of
multicore computers. Hirzel et al.\ introduced the \textsf{SPL}
language~\cite{hirzel_schneider_gedik_2017} as part of the \textsf{IBM~Streams}
product in~2010. Figure~\ref{fig:spl} shows an example for a similar
use-case as Figure~\ref{fig:cql}. Line~1 defines a stream
\lstinline{Calls} by invoking an operator \lstinline{CallsSrc}, and
\mbox{Lines 3-8} define a stream \lstinline{Stats} by invoking an
operator \lstinline{Aggregate}. An SPL program explicitly specifies a
directed graph of stream edges and operator nodes. Streams carry
tuples; in the examples, tuple attributes contain primitive values,
but in general, they can also contain compound values such as other
tuples or lists.  Operators create and transform streams; operators
are defined by users or libraries, not built into the
language. Operators can be further configured upon invocation, for
example, with windows or output assignments. To facilitate
distribution, SPL's semantics are defined to require minimal
synchronization between operators~\cite{soule_et_al_2016}.

One can
view Borealis as the evolutionary link between relational streaming
and big-data streaming languages~\cite{abadi_et_al_2005}.
In the big-data domain, SPL has been followed by
\textsf{MillWheel}~\cite{akidau_et_al_2013}, \textsf{Spark
Streaming}~\cite{zaharia_et_al_2013},
\textsf{Storm}~\cite{toshniwal_et_al_2014}, \textsf{Flink}~\cite{carbone_et_al_2015},
and others.  Unlike SPL, which is a stand-alone language, these
systems offer languages that are embedded in a general-purpose host
language, typically Java. An embedded language is essentially an
advanced library or framework that makes heavy use of host-language
abstractions such as lambdas, generics, and local variable type
inference. Embedded languages benefit from simple interoperability
with their host language. Also, they can leverage tools and skills
from the host language~\cite{hudak_1998}. On the downside, since they
are not self-contained, it is hard to isolate clear core semantics
without pulling in the host language. This inhibits debugging,
optimization, and standardization.
