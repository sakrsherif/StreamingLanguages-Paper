\subsection{Big-Data Streaming}\label{sec:big} % Martin

\begin{figure}[!h]
\begin{lstlisting}
stream<float64 len, rstring caller> Calls = CallsSrc() {}
type StatsT = tuple<float64 len, int32 num, rstring who>;
stream<StatsT> StatsS = Aggregate(Calls) {
   window Calls: sliding, time(24.0*60.0*60.0), time(60.0);
   output StatsS: len = Max(Calls.len),
                  num = MaxCount(Calls.len),
                  who = ArgMax(Calls.len, Calls.caller);
}
\end{lstlisting}
\vspace*{-4mm}
\caption{\label{fig:spl}SPL code example.}
\end{figure}

The need to handle diverse data and processing needs at scale
motivated several recent big-data streaming languages and
systems~\cite{carbone_et_al_2015,hirzel_schneider_gedik_2017,toshniwal_et_al_2014,zaharia_et_al_2013}.
Each of them makes it easy to integrate operators written in
general-purpose languages and to parallelize them on clusters of
multicore computers. Hirzel et al.\ introduced the SPL
language~\cite{hirzel_schneider_gedik_2017} as part of the IBM~Streams
product in~2010. Figure~\ref{fig:spl} shows an example for a similar
use-case as Figure~\ref{fig:cql}. Line~1 defines a stream
\lstinline{Calls} by invoking an operator \lstinline{CallsSrc}, and
\mbox{Lines 3-8} define a stream \lstinline{StatsS} by invoking an
operator \lstinline{Aggregate}. An SPL program explicitly specifies a
directed graph of stream edges and operator nodes. Streams carry
tuples; in the examples, tuple attributes contain primitive values,
but in general, they can also contain compound values such as other
tuples or lists.  Operators create and transform streams; operators
are defined by users or libraries, not built into the
language. Operators can be further configured upon invocation, for
example, with windows or output assignments. To facilitate
distribution, SPL's semantics are defined to require minimal
synchronization between operators~\cite{soule_et_al_2016}. One can
view Borealis as the evolutionary link between relational streaming
and big-data streaming languages~\cite{abadi_et_al_2005}. In the
big-data domain, SPL has been followed by
Storm~\cite{toshniwal_et_al_2014}, Spark
Streaming~\cite{zaharia_et_al_2013}, Flink~\cite{carbone_et_al_2015},
and others.

\begin{alltt}TODO\scriptsize Martin
- more extensively describe Storm, Spark, and Flink
\end{alltt}
