\subsection{XML Streaming}\label{sec:xml} % Sherif

In 2002, Diao et al.~\cite{diao_et_al_2002,diao2003high} presented
\textsf{YFilter} as a system that implemented continuous queries over XML
streaming data using a subset of the XPath
language~\cite{clark_derose_1999}. YFilter applied a multi-query
optimization that used a single finite state machine to represent and
evaluate several XPath expressions. In particular, YFilter exploited
the commonality among path queries by merging the common prefixes of
the paths so that they were processed at most once. This shared
processing mechanism provided significant performance improvement by
avoiding redundant processing for duplicate path expressions.  To
handle value-based predicates that address contents of elements,
YFilter applied two alternative approaches. The first approach
evaluates the predicates once the addressed elements are read from a
document, while the second approach postpones predicate evaluation
until the corresponding path expression has been entirely matched.

Before YFilter, which processed streams of XML documents, came
\textsf{NiagaraCQ}, which processed update streams to existing XML
documents~\cite{chen_et_al_2000}, borrowing syntax from
XML-QL~\cite{deutsch1999query}.  NiagaraCQ supported incremental
evaluation to consider only the changed portion of each updated XML
file. It supported two kinds of continuous queries:
\emph{change-based} queries, which trigger as soon as new relevant
data becomes available, and \emph{timer-based} queries, which trigger
only at specified time intervals.

After YFilter, various other languages processed streams of XML
documents.  For instance, \textsf{XSQ}~\cite{peng_chawathe_2003}, which was
also based on XPath, used a hierarchical arrangement of pushdown
transducers augmented with buffers. Besides filtering, XSQ had output
functions including aggregations.  Mendell et al.\ extended a big-data
streaming language, SPL~\cite{hirzel_schneider_gedik_2017}, with a new
operator for XML stream processing,
\emph{XMLParse}~\cite{mendell_et_al_2012}. It allows the user to not
only filter the XML stream but also to transform it to a data format
amenable for downstream processing.
