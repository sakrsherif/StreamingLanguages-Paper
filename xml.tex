\subsection{XML Streaming}\label{sec:xml} % Sherif

\begin{figure}[!h]
\begin{lstlisting}
CREATE ~\textit{cq\_name}~
         ~\textit{xml\_ql\_query}~
     DO ~\textit{action}~
  {START ~\textit{start\_time}~}
  {EVERY ~\textit{time\_interval}~}
  {EXPIRE ~\textit{expiration\_time}~}
\end{lstlisting}
\vspace*{-4mm}
\caption{\label{fig:Niagra}NiagaraCQ code example.}
\end{figure}

Chen et al.\ introduced NiagaraCQ~\cite{chen_et_al_2000} in 2000 as a
continuous query sub-system of the Niagara internet query engine, a
data\-base for querying distributed XML data sets developed at
University of Wisconsin and Oregon Graduate
Institute~\cite{naughton2001niagara}. NiagaraCQ implements continuous
query processing over XML files by supporting incremental evaluation
and considering only the changed portion of each updated XML file. It
supports two types of continuous queries that differ in the criteria
triggering their execution: \emph{change-based} queries, which trigger
as soon as new relevant data becomes available, and \emph{timer-based}
queries, which trigger only at specified time intervals.  NiagaraCQ is
based on XML-QL~\cite{deutsch1999query}.  It provides a command
language for creating continuous queries that follows the form
illustrated in Figure~\ref{fig:Niagra}. Using this command language,
users can implement continuous queries that combines an ordinary
XML-QL query with additional time information.  The
\textsf{\small\textit{xml\_ql\_query}} becomes effective at the
defined \textsf{\small\textit{start\_time}}.  The
\textsf{\small\textit{time\_interval}} indicates how often the query
will be executed. A query is timer-based if its
\textsf{\small\textit{time\_interval}} is non-zero; otherwise, it is
change-based.  The continuous query will be deactivated after its
\textsf{\small\textit{expiration\_time}}. The
\textsf{\small\textit{action}} triggers once the results of the XML-QL
query expression is returned.  A key optimization mechanism used in
the NiagaraCQ system is grouping the execution of similar queries to
minimize redundant work.

YFilter~\cite{diao_et_al_2002,diao2003high} implements
continuous queries based on XPath~\cite{clark_derose_1999}, also with
multi-query optimization. The main idea of YFilter is to use a single
finite state machine to represent and evaluate several XPath
expressions. In particular, YFilter exploits the commonality among path queries by merging the common prefixes
of the paths so that they are processed at most once. This  shared processing mechanism provides significant performance
improvement by avoiding redundant processing for duplicate path expressions. In order to handle value-based predicates that address contents of elements, YFilter applies two alternative approaches. The first approach approach evaluates the  predicates once the addressed elements are read from a document, while the other approach postpones predicate evaluation until the corresponding path expression has been entirely matched.
